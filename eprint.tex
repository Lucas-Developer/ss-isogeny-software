\documentclass[envcountsect,envcountsame,runningheads]{llncs}   % [twoside]
%\usepackage{fancyheadings,amsmath,amssymb,amsfonts,color,tikz,pgflibraryplotmarks,verbatim}
\usepackage{amsmath,amssymb,amsfonts,color,verbatim,tikz,pgflibraryplotmarks,url}
\usepackage[full-version]{optional}
\opt{full-version}{
\usepackage[letterpaper,hmargin=1in,vmargin=1.25in]{geometry}
}
\usepackage{algorithm}
\usepackage{algorithmic}
\renewcommand{\algorithmicrequire}{\textbf{Input:}}
\renewcommand{\algorithmicensure}{\textbf{Output:}}
\def\aa//{{\raise0.5pt\hbox{\tt/\kern-1.5pt/}}}

\newtheorem{prop}[theorem]{Proposition}

\newcommand{\QQ}{{\mathbb{Q}}}
\newcommand{\ZZ}{{\mathbb{Z}}}
\newcommand{\RR}{{\mathbb{R}}}
\newcommand{\CC}{{\mathbb{C}}}
\newcommand{\FF}{{\mathbb{F}}}
\newcommand{\OO}{{\mathcal{O}}}
\newcommand{\VV}{{\mathcal{V}}}
\newcommand{\EE}{{\mathcal{E}}}
\newcommand{\iso}{\cong}
\newcommand{\id}{\operatorname{sID}}
\newcommand{\cyc}[1]{{\langle #1 \rangle}}

\newcommand{\fixme}[1]{{\color{blue} \sf $\clubsuit\clubsuit\clubsuit$ FIXME: [#1]}}

%\newtheorem{problem}[theorem]{Problem}

\title{Towards quantum-resistant cryptosystems
from supersingular elliptic curve isogenies}
\titlerunning{Towards quantum-resistant cryptosystems from isogenies}
\author{David Jao\inst{1} \and Luca De Feo\inst{2}}
\institute{
Department of Combinatorics and Optimization \\
University of Waterloo, Waterloo, Ontario, N2L 3G1, Canada\\
\email{djao@math.uwaterloo.ca}\\[\baselineskip]
\and
Laboratoire PRiSM\\
Universit\'e de Versailles, 78035 Versailles, France\\
\email{http://www.prism.uvsq.fr/\homedir dfl}
}

\begin{document}

\maketitle 
 
\begin{abstract}
We present new candidates for quantum-resistant public-key
cryptosystems based on the conjectured difficulty of finding isogenies
between supersingular elliptic curves.  The main technical idea in our
scheme is that we transmit the images of torsion bases under the isogeny
in order to allow the two parties to arrive at a common shared key
despite the noncommutativity of the endomorphism ring.  Our work is
motivated by the recent development of a subexponential-time quantum
algorithm for constructing isogenies between ordinary elliptic curves.
In the supersingular case, by contrast, the fastest known quantum attack
remains exponential, since the noncommutativity of the endomorphism ring
means that the approach used in the ordinary case does not apply. We
give a precise formulation of the necessary computational assumption
along with a discussion of its validity\opt{full-version}{, and prove the
security of our
protocols under this assumption}. In addition, we present implementation
results showing that our protocols are multiple orders of magnitude faster
than previous isogeny-based cryptosystems over ordinary curves.

\ 

\textbf{Keywords:} elliptic curves, isogenies, quantum-resistant public-key cryptosystems
\end{abstract}

%%%%%%%%%%%%%%%%%%%%%%%%%%%%%
\section{Introduction}
%%%%%%%%%%%%%%%%%%%%%%%%%%%%%

The Diffie-Hellman scheme is a fundamental protocol for public-key
exchange between two parties. Its original definition over finite
fields is based on the hardness of computing the map $g,g^a,g^b
\mapsto g^{ab}$ for $g \in \FF_p^*$, while its elliptic curve analogue
depends on the difficulty of computing $P,aP,bP \mapsto abP$ for
points $P$ on an elliptic curve. Recently, Stolbunov~\cite{Stol}
proposed a Diffie-Hellman type system based on the difficulty of
computing isogenies between ordinary elliptic curves, with the stated
aim of obtaining quantum-resistant cryptographic protocols.  The
fastest known (classical) probabilistic algorithm for solving this
problem is the algorithm of Galbraith and Stolbunov~\cite{gs}, based
on the algorithm of Galbraith, Hess, and Smart~\cite{GHS}. This
algorithm is exponential, with a worst-case running time of
$O(\sqrt[4]{q})$. However, on a quantum computer, recent work of
Childs et al.~\cite{CJS} has shown that the private keys in Stolbunov's
system can be recovered in subexponential time. Moreover, even if
we only consider classical attacks in assessing security levels, Stolbunov's
scheme requires 229 seconds (even with precomputation) to perform a
single key exchange operation at the 128-bit security level on a
desktop PC~\cite[Table 1]{Stol}.

In this work we present isogeny-based cryptosystems that address both
the performance and security drawbacks of Stolbunov's system. Our
scheme achieves performance on the order of one second
(cf. Section~\ref{sec:imp}) at the 128-bit security level (as measured
against the fastest known quantum attacks) using
desktop PCs, making it far faster than Stolbunov's scheme. In terms of
security, our scheme is not vulnerable to the algorithm of Childs et
al.~\cite{CJS}, nor to any algorithm of this type, since it is not
based on a group action. The fastest known attacks against our scheme,
even on quantum computers, require fully exponential time. Our scheme
involves a new computational assumption upon which its quantum resistance
is based, and like all new computational assumptions,
further study and the passage of time is needed for
validation. Nevertheless, we believe our proposal represents
a promising candidate for quantum-resistant isogeny-based public-key
cryptography.

Our proposal uses isogenies between \emph{supersingular} elliptic
curves rather than ordinary elliptic curves. The main technical
difficulty is that, in the supersingular case, the endomorphism ring
is noncommutative, whereas Diffie-Hellman type protocols require
commutativity. We show how to overcome this obstacle by providing the
outputs of the isogeny on certain points as auxiliary input to the
protocol. To the best of our knowledge, nothing
similar to this idea has ever previously appeared in the
literature. Providing this auxiliary input does not seem to
make the problem of finding isogenies any easier; see
Section~\ref{subsec:hardness} for a full discussion. The multiple
orders of magnitude of performance gains in our scheme arise from the
fact that supersingular isogeny graphs are much faster to navigate
than ordinary graphs. In Section~\ref{subsec:proof} we provide formal
statements of the hardness assumptions and security reductions for our
system. Finally, in Section~\ref{sec:imp} we present implementation
results confirming the correctness and performance of our protocol.


%%%%%%%%%%%%%%%%%%%%%%%%%%%%%
\section{Isogenies}\label{isogenies}
%%%%%%%%%%%%%%%%%%%%%%%%%%%%%
Let $E_1$ and $E_2$ be elliptic curves defined over a finite field
$\FF_q$.
An isogeny $\phi: E_1 \rightarrow E_2$ defined over $\FF_q$ is a
non-constant rational map defined over $\FF_q$ which is also a group
homomorphism from $E_1(\FF_q)$ to $E_2(\FF_q)$ \cite[III.4]{Sil}. The
degree of an isogeny is its degree as a rational map.  For separable
isogenies, to have degree $\ell$ means to have kernel of size $\ell$.
Every isogeny of degree greater than 1 can be factored into a
composition of isogenies of prime degree over $\bar{\FF}_q$
\cite{Couv}.

An endomorphism of an elliptic curve $E$ defined over $\FF_q$ is an
isogeny $E \rightarrow E$ defined over $\FF_{q^m}$ for some $m$. The
set of endomorphisms of $E$ together with the zero map forms a ring
under the operations of pointwise addition and composition; this ring
is called the endomorphism ring of $E$ and denoted End($E$). The ring
End($E$) is isomorphic either to an order in a quaternion algebra or
to an order in an imaginary quadratic field \cite[V.3.1]{Sil}; in the
first case we say $E$ is supersingular and in the second case we say
$E$ is ordinary.

Two elliptic curves $E_1$ and $E_2$ defined over $\FF_q$ are said to
be isogenous over $\FF_q$ if there exists an isogeny $\phi\colon E_1
\to E_2$ defined over $\FF_q$. A theorem of Tate states that
two curves $E_1$ and $E_2$ are isogenous over $\FF_q$ if and only if
$\#E_1(\FF_q) = \#E_2(\FF_q)$ \cite[$\S$3]{Tate}. Since every isogeny
has a dual isogeny \cite[III.6.1]{Sil}, the property of being
isogenous over $\FF_q$ is an equivalence relation on the finite set of
$\bar{\FF}_q$-isomorphism classes of elliptic curves defined over
$\FF_q$.  Accordingly, we define an isogeny class to be an equivalence
class of elliptic curves, taken up to $\bar{\FF}_q$-isomorphism, under
this equivalence relation.

The $\ell$-torsion group of $E$, denoted $E[\ell]$, is the set of all
points $P \in E(\bar{\FF}_q)$ such that $\ell P$ is the identity. For
$\ell$ such that $p\nmid \ell$, we have $E[\ell] \cong \ZZ/\ell\ZZ \oplus
\ZZ/\ell\ZZ.$

Curves in the same isogeny class are either all supersingular or all
ordinary. Traditionally, most elliptic curve cryptography uses
ordinary curves; however, for this work we will be interested in
supersingular curves. We assume for the remainder of this paper that
we are in the supersingular case.

Supersingular curves are all defined over $\FF_{p^2}$, and for every
prime $\ell \nmid p$, there exist $\ell+1$ isogenies (counting
multiplicities) of degree $\ell$ originating from any given such
supersingular curve.
Given an elliptic curve $E$ and a finite subgroup $\Phi$ of $E$, there
is up to isomorphism a unique isogeny $E \to E'$ having kernel
$\Phi$~\cite[III.4.12]{Sil}. Hence we can identify an isogeny by
specifying its kernel, and conversely given a kernel subgroup the
corresponding isogeny can be computed using V\'elu's
formulas~\cite{Velu}. Typically, this correspondence is of little use,
since the kernel, or any representation thereof, is usually as
unwieldy as the isogeny itself. However, in the special case of
kernels generated by $\FF_{p^2}$-rational points of smooth order,
specifying a generator of the kernel allows for compact representation
and efficient computation of the corresponding isogeny, as we
demonstrate below.

\opt{full-version}{
%%%%%%%%%%%%%%%%%%%%%%%%%%%%%
\subsection{Ramanujan graphs}\label{ram_graph} 

Let $G = (\mathcal{V},\EE)$ be a finite graph on $h$ vertices $\VV$
with undirected edges $\EE$.  Suppose $G$ is a regular graph of degree
$k$, i.e., exactly $k$ edges meet at each vertex. Given a
labeling of the vertices $\VV = \{v_1,\ldots , v_h\}$, the adjacency
matrix of $G$ is the symmetric $h\times h$ matrix $A$ whose $ij$-th
entry $A_{i,j} = 1$ if an edge exists between $v_i$ and $v_j$ and 0
otherwise.

It is convenient to identify functions on $\VV$ with vectors in
$\RR^h$ via this labeling, and therefore also think of $A$ as a
self-adjoint operator on $L^2(\VV)$.  All of the eigenvalues of $A$
satisfy the bound $|\lambda| \leq k$. Constant vectors are
eigenfunctions of $A$ with eigenvalue $k$, which for obvious reasons
is called the trivial eigenvalue $\lambda_{\operatorname{triv}}$. A
family of such graphs $G$ with $h \rightarrow \infty$ is said to be a
sequence of {\it expander graphs} if all other eigenvalues of their
adjacency matrices are bounded away from
$\lambda_{\operatorname{triv}}= k$ by a fixed
amount.\footnote{Expansion is usually phrased in terms of the number
  of neighbors of subsets of $G$, but the spectral condition here is
  equivalent for $k$-regular graphs and also more useful for our
  purposes.}  In particular, no other eigenvalue is equal to $k$; this
implies the graph is connected.  A Ramanujan graph is a special type
of expander which has $|\lambda| \leq 2\sqrt{k-1}$ for any nontrivial
eigenvalue which is not equal to $-k$ (this last possibility happens
if and only if the graph is bipartite). The Ramanujan property was
first defined in \cite{LubPS}. It characterizes the optimal separation
between the two largest eigenvalues of the graph adjacency matrix, and
implies the expansion property.

A fundamental use of expanders is to prove the rapid mixing of the
random walk on $\VV$ along the edges $\EE$. The following rapid mixing
result is standard but we present it below for completeness. For the
proof, see~\cite{JMV} or \cite{DSV,Lub,Sarnak}.

\begin{prop}\label{prop:mixing} Let $G$ be a regular graph of degree
  $k$ on $h$ vertices. Suppose that the eigenvalue $\lambda$ of any
  nonconstant eigenvector satisfies the bound $|\lambda| \leq c$ for
  some $c < k$. Let $S$ be any subset of the vertices of $G$, and $x$
  be any vertex in $G$. Then a random walk of length at least
  $\frac{\log 2h/|S|^{1/2}}{\log k/c}$ starting from $x$ will land in
  $S$ with probability at least $\frac{|S|}{2h} = \frac{|S|}{2|G|}$.
\end{prop}

%%%%%%%%%%%%%%%%%%%%%%%%%%%%%
\subsection{Isogeny graphs}\label{isog_graph} 

An isogeny graph is a graph whose nodes consist of all elliptic curves
in $\FF_q$ belonging to a fixed isogeny class, up to
$\bar{\FF}_q$-isomorphism (so that two elliptic curves which are
isomorphic over $\bar{\FF}_q$ represent the same node in the
graph). In practice, the nodes are represented using $j$-invariants,
which are invariant up to isomorphism. 
Isogeny graphs for supersingular elliptic curves were first considered
by Mestre \cite{Mestre}, and were shown by Pizer \cite{pizer1,pizer2}
to have the Ramanujan property.

Every supersingular elliptic curve in characteristic $p$ is defined
over either $\FF_p$ or $\FF_{p^2}$ \cite{Sil}, so it suffices to fix
$\FF_q = \FF_{p^2}$ as the field of definition for this
discussion. Thus, in contrast to ordinary curves, there are a finite
number of isomorphism classes of curves in any given isogeny class;
this number is in fact the genus $g$ of the modular curve $X_0(p)$,
which is roughly $\frac{p+1}{12}$. It turns out that all supersingular
curves defined over $\FF_{p^2}$ belong to the same isogeny
class~\cite{Mestre}. For a fixed prime value of $\ell \neq p$, we
define the vertices of the supersingular isogeny graph $\mathcal{G}$
to consist of these $g$ isomorphism classes of curves, with
edges given by isomorphism classes of degree-$\ell$ isogenies,
defined as follows: two isogenies $\phi_1, \phi_2 \colon E_i \to E_j$
are isomorphic if there exists an automorphism $\alpha \in
\text{Aut}(E_j)$ (i.e., an invertible endomorphism) such that $\phi_2
= \alpha\phi_1$. Pizer~\cite{pizer1,pizer2} has shown that
$\mathcal{G}$ is a connected $k = \ell + 1$-regular multigraph
satisfying the Ramanujan bound of $|\lambda| \leq 2\sqrt{\ell} =
2\sqrt{k - 1}$ for the nontrivial eigenvalues of its adjacency matrix.
}

%%%%%%%%%%%%%%%%%%%%%%%%%%%%%
\section{Public-key cryptosystems based on supersingular curves}\label{sec:kep}
%%%%%%%%%%%%%%%%%%%%%%%%%%%%%

In this section we present a key-exchange protocol and a public-key
cryptosystem analogous to those of \cite{R&S,Stol}, using
supersingular elliptic curves.  Since the discrete logarithm problem
is unimportant when elliptic curves are used in an isogeny-based
system, we propose using supersingular curves of smooth order to
improve performance. In the supersingular setting, it is easy to
construct curves of smooth order, and using a smooth order curve will
give a large number of isogenies that are fast to
compute. Specifically, we fix $\FF_q = \FF_{p^2}$ as the field of
definition, where $p$ is a prime of the form $\ell_A^{e_A}
\ell_B^{e_B}\cdot f \pm 1$.  Here $\ell_A$ and $\ell_B$ are small
primes, and $f$ is a cofactor such that $p$ is prime. Alice and Bob
will each take a random walk on a different isogeny graph; Alice will
use the graph consisting of isogenies of degrees $\ell_A$, and Bob
will use the graph of degree $\ell_B$ isogenies. The main technical
modification is that, since ideal classes no longer commute (or indeed
even multiply together) in the supersingular case, extra information
must be communicated as part of the protocol in order to ensure that
both parties arrive at the same common value. This is in contrast to
the ordinary case~\cite{Stol}, where the existence of an abelian class
group allows for the straightforward creation of a Diffie-Hellman type
system.

\begin{figure}[t]
\begin{center} 
\begin{tabular}{l c l}
\cline{1-1} \cline{3-3} 
$\mathcal{A}$ & & $\mathcal{B}$ \\ 
\cline{1-1} \cline{3-3} 
\textbf{Input:} $A,B,\id$ & & \textbf{Input:} $B$ \\
$m_A,n_A \in_{R} \ZZ/\ell_A^{e_A}\ZZ$ & & $m_B,n_B \in_{R} \ZZ/\ell_B^{e_B}\ZZ$ \\
$\phi_A := E_0/\cyc{[m_A]P_A+[n_A]Q_A }$ & & $\phi_B :=
E_0/\cyc{[m_B]P_B+[n_B]Q_B} $ \\ 
 & $\xrightarrow[]{\substack{A, \id \\ \phi_A(P_B),\\ \phi_A(Q_B),\\ E_A}} $ &  \\
  & $\xleftarrow[]{\substack{B, \id \\ \phi_B(P_A),\\ \phi_B(Q_A),\\ E_B}} $ &  \\
$E_{AB} := $ &  & $ E_{BA} := $ \\
$E_B/\scriptstyle{\cyc{[m_A]\phi_B(P_A)+[n_A]\phi_B(Q_A)}} $ &  & $
E_A/\scriptstyle{\cyc{[m_B]\phi_A(P_B)+[n_B]\phi_A(Q_B)}}$ \\
\textbf{Output:} $j(E_{AB}), \id$ & & \textbf{Output:} $j(E_{BA}), \id$

\end{tabular}
\end{center}

\begin{center}
\begin{tikzpicture}
\node (1) at (-5cm,0cm){$E_0$};

\node (2) at (0cm,2cm){$E_A$};
\draw [color = red, ->] (1) -- (2);
\path (1) -- (2) node [above,sloped,pos=0.5]{$\scriptscriptstyle ker(\phi_A)=\cyc{[m_A]P_A+[n_A]Q_A}$};
\path (1) -- (2) node [below,sloped,pos=0.5]{$\scriptscriptstyle\phi_A(P_B),\phi_A(Q_B)$};

\node (3) at (0cm,-2cm){$E_B$};
\draw [color = blue, ->] (1) -- (3);
\path (1) -- (3) node [below,sloped,pos=0.5]{$\scriptscriptstyle ker(\phi_B)=\cyc{[m_B]P_B+[n_B]Q_B}$};
\path (1) -- (3) node [above,sloped,pos=0.5]{$\scriptscriptstyle\phi_B(P_A),\phi_B(Q_A)$};

\node (2a) at (0cm,1.9cm){};
\node (2b) at (-.1cm,1.9cm){};

\node (3a) at (.1cm,-1.9cm){};
\node (3b) at (0cm,-1.9cm){};

\draw [color = red, dashed, ->] (2a) -- (3a);
\draw [color = blue, dashed, <-] (2b) -- (3b);

\node (4) at (5cm,-.4cm){$E_{AB}$};
\draw [color = red, ->] (3) -- (4);
\path (3) -- (4) node [below,sloped,pos=0.5]{$\scriptscriptstyle ker(\phi_A')=\cyc{[m_A]\phi_B(P_A)+[n_A]\phi_B(Q_A)}$};

\node (5) at (5cm,.4cm){$E_{BA}$};
\draw [color = blue, ->] (2) -- (5);
\path (2) -- (5) node [above,sloped,pos=0.5]{$\scriptscriptstyle ker(\phi_B')=\cyc{[m_B]\phi_A(P_B)+[n_B]\phi_A(Q_B)}$};

\node (6) at (5cm,0cm){$\|$};

\end{tikzpicture}
\end{center}
\caption{Key-exchange protocol using isogenies on supersingular
  curves.}
\label{fig:kep}
\end{figure}

\subsection{Key exchange}\label{subsec:kep}

We fix as public parameters a supersingular curve $E_0$ defined over
$\FF_{p^2}$, and bases $\{P_{A},Q_{A}\}$ and $\{P_{B},Q_{B}\}$ which
generate $E_0[\ell_A^{e_A}]$ and $E_0[\ell_B^{e_B}]$ respectively, so
that $\cyc{P_{A}, Q_{A}} = E_0[\ell_A^{e_A}]$ and $\cyc{P_{B}, Q_{B}}
= E_0[\ell_B^{e_B}]$.  Alice chooses two random elements $m_A,n_A
\in_R \ZZ/\ell_A^{e_A}\ZZ$, not both divisible by $\ell_A$, and computes
an isogeny $\phi_A\colon E_0 \to E_A$ with kernel $K_A :=
\cyc{[m_A]P_A+[n_A]Q_A}$. Alice also computes the image
$\{\phi_A(P_B), \phi_A(Q_B)\} \subset E_A$ of the basis
$\{P_{B},Q_{B}\}$ for $E_0[\ell_B^{e_B}]$ under her secret isogeny
$\phi_A$, and sends these points to Bob together with
$E_A$. Similarly, Bob selects random elements $m_B,n_B \in_R
\ZZ/\ell_B^{e_B}\ZZ$ and computes an isogeny $\phi_B\colon E_0 \to E_B$
having kernel $K_B := \cyc{[m_B]P_B+[n_B]Q_B}$, along with the points
$\{\phi_B(P_A), \phi_B(Q_A)\}$. Upon receipt of $E_B$
and $\phi_B(P_A),\phi_B(Q_A) \in E_B$ from Bob, Alice computes an
isogeny $\phi_A' \colon E_B \to E_{AB}$ having kernel equal to
$\cyc{[m_A]\phi_B(P_A)+[n_A]\phi_B(Q_A)}$; Bob proceeds \emph{mutatis
mutandis}.  Alice and Bob can then use the common $j$-invariant of
\[ E_{AB} = \phi_B'(\phi_A(E_0))=  \phi_A'(\phi_B(E_0)) =
E_0/\scriptstyle{\cyc{[m_A]P_A+[n_A]Q_A ,[m_B]P_B+[n_B]Q_B }} \]
to form a secret shared key. For specific details of how each of the
above computations can be performed efficiently, we refer the reader
to Section~\ref{sec:alg}.

The full protocol is given in Figure~\ref{fig:kep}. We denote by $A$
and $B$ the identifiers of Alice and Bob, and use $\id$ to denote the
unique session identifier.

\subsection{Public-key encryption}\label{subsec:pk}

The key-exchange protocol of Section~\ref{subsec:kep} can easily be
adapted to yield a public-key cryptosystem, in much the same way that
Elgamal encryption follows from Diffie-Hellman. We briefly give the
details here. All notation is the same as above. Stolbunov~\cite{Stol}
uses a similar construction, upon which ours is based.

\begin{description}
\item[Setup:] Choose $p = \ell_A^{e_A} \ell_B^{e_B} \cdot f \pm 1$,
  $E_0$, $\{P_A,Q_A\}$, $\{P_B,Q_B\}$ as above. Let $\mathcal{H} =
  \{H_k: k \in K\}$ be a hash function family indexed by a finite set
  $K$, where each $H_k$ is a function from $\FF_{p^2}$ to the message
  space $\{0,1\}^w$.
\item[Key generation.] Choose two random elements $m_A,n_A \in_R
\ZZ/\ell_A^{e_A}\ZZ$, not both divisible by $\ell_A$. Compute $E_A,
\phi_A(P_B), \phi_A(Q_B)$ as above, and choose a random element $k
\in_R K$. The public key is the tuple $(E_A, \phi_A(P_B), \phi_A(Q_B), k)$ and
the private key is $(m_A,n_A,k)$.
\item[Encryption.] Given a public key $(E_A, \phi_A(P_B), \phi_A(Q_B),
  k)$ and a message $m \in \{0,1\}^w$, choose two random elements
  $m_B,n_B \in_R \ZZ/\ell_B^{e_B}\ZZ$, not both divisible by $\ell_B$,
  and compute
\begin{align*}
h &= H_k(j(E_{AB})), \\
c &= h \oplus m.
\end{align*}
The ciphertext is $(E_B, \phi_B(P_A), \phi_B(Q_A), c)$.
\item[Decryption.] Given a ciphertext $(E_B, \phi_B(P_A), \phi_B(Q_A),
  c)$ and a private key $(m_A,n_A,k)$, compute the $j$-invariant
  $j(E_{AB})$ and set
\begin{align*}
h &= H_k(j(E_{AB})), \\
m &= h \oplus c.
\end{align*}
The plaintext is $m$.
\end{description}

\section{Algorithmic aspects}\label{sec:alg}

We now give specific algorithms to implement the abovementioned steps
efficiently. 

\subsection{Parameter generation}\label{subsec:parameter}

For any fixed choice of $\ell_A^{e_A}$ and $\ell_B^{e_B}$, one can
easily test random values of $f$ (of any desired cryptographic size)
until a value is found for which $p=\ell_A^{e_A} \ell_B^{e_B}\cdot f -
1$ or $p=\ell_A^{e_A} \ell_B^{e_B}\cdot f + 1$ is prime; the prime
number theorem in arithmetic progressions (specifically, the effective
version of Lagarias and Odlyzko~\cite{lo}) provides a sufficient lower
bound for the density of such primes. 

Once the prime $p= \ell_A^{e_A} \ell_B^{e_B}\cdot f \pm 1$ is known,
Br\"oker~\cite{broker-ss} has shown that it is easy to find a
supersingular curve $E$ over $\FF_{p^2}$ having cardinality $(p\mp
1)^2 = (\ell_A^{e_A} \ell_B^{e_B}\cdot f)^2$. Starting from $E$, one
can select a random supersingular curve $E_0$ over $\FF_{p^2}$ by
means of random walks on the isogeny graph\opt{full-version}{
(cf. Proposition~\ref{prop:mixing})};
alternatively, one can simply take $E_0 = E$. In either case, $E_0$
has group structure $(\ZZ/(p\mp 1)\ZZ)^2$. To find a basis for
$E_0[\ell_A^{e_A}]$, choose a random point $P \in_R E_0(\FF_{p^2})$
and multiply it by $(\ell_B^{e_B}\cdot f)^2$ to obtain a point $P'$ of
order dividing $\ell_A^{e_A}$. With high probability, $P'$ will have
order exactly $\ell_A^{e_A}$; one can of course check this by
multiplying $P'$ by powers of $\ell_A$. If the check succeeds, then
set $P_A = P'$; otherwise try again with another $P$. A second point
$Q_A$ of order $\ell_A^{e_A}$ can be obtained in the same way.  To
check whether $Q_A$ is independent of $P_A$, simply compute the Weil
pairing $e(P_A,Q_A)$ in $E[\ell_A^{e_A}]$ and check that the result
has order $\ell_A^{e_A}$; as before, this happens with high
probability, and if not, just choose another point $Q_A$. Note that
the choice of basis has no effect on the security of the scheme, since
one can convert from one basis to another using extended discrete
logarithms, which are easy to compute in $E_0[\ell_A^{e_A}]$
by~\cite{teske-ph}.

\subsection{Key exchange}\label{subsec:kea}

It remains to describe how Alice and Bob can compute isogenies of a
given kernel.  We show how to compute
$\phi_A\colon E_0 \to E_A$ where $E_A = E_0/\cyc{[m_A]P_A +
  [n_A]Q_A}$; the same procedure suffices to compute all the other
isogenies mentioned. The computation is performed using a version of
Hensel lifting modulo $\ell_A$. Let $R_0 := [m_A]P_A + [n_A]Q_A$. 
The order of $R_0$ is $\ell_A^{e_A}$. For $0\le i <e_A$, let
\begin{equation*}
  E_{i+1} = E_i/\cyc{\ell_A^{e_A-i-1}R_i},\qquad
  \phi_i : E_i \to E_{i+1},\qquad
  R_{i+1} = \phi_i(R_i),
\end{equation*}
where $\phi_i$ is a degree $\ell_A$ isogeny from $E_i$ to $E_{i+1}$.
Then $E_A = E_{e_A}$ and
$\phi_A=\phi_{e_A-1}\circ\cdots\circ\phi_0$.

Figure~\ref{fig:kea} gives two algorithms for this task. They both
compute iteratively $(R_i,\ell_A^{e_A-i-1}R_i,\phi_i,E_{i+1})$ for
$i<e_A$, but they differ in the strategy. The first one, which we will
refer to as \emph{multiplication-oriented}, computes at each iteration
$\ell_A^{e_A-i-1}R_i$ from $R_i$ using point addition (or duplication,
or triplication). The second one, which we call
\emph{isogeny-oriented}, first forms the list $(\ell_A^jR_0)_{j<e_A}$
using point addition, then at each iteration computes the list
$(\ell_A^jR_{i+1})_{j<e_A-i-1}$ by evaluating $\phi_i(\ell_A^jR_i)$
for each $j$. Observe that Alice and Bob can use one algorithm or the
other independently.

\begin{figure}[t]
  \centering

  \begin{minipage}[t]{0.47\linewidth}
    \textbf{Multiplication based}
    \hrule\smallskip
    \begin{algorithmic}[1]
      \REQUIRE $E_0, R_0$
      \FOR{$0\le i<e_A$}
      \STATE\label{alg:mul:mul} $P_i \leftarrow \ell_A^{e_a-i-1}R_0$;
      \STATE\label{alg:mul:velu} Compute $\phi_i : E_i \to E_i/\cyc{P_i}$;
      \STATE $E_{i+1} \leftarrow E_i/\cyc{P_i}$;
      \STATE\label{alg:mul:eval} $R_{i+1} \leftarrow \phi_i(R_i)$;
      \ENDFOR
      \ENSURE $E_{e_A}$
    \end{algorithmic}
  \end{minipage}
  %%
  \hfill
  %%
  \begin{minipage}[t]{0.47\linewidth}
    \textbf{Isogeny based}
    \hrule\smallskip
    \begin{algorithmic}[1]
      \REQUIRE $E_0, R_0$
      \STATE $Q_0 \leftarrow R_0$;
      \FOR{$0\le j<e_A-1$}
      \STATE $Q_{j+1} \leftarrow \ell_AQ_j$;
      \ENDFOR
      \FOR{$0\le i<e_A$}
      \STATE Compute $\phi_i : E_i \to E_i/\cyc{Q_{e_A-1}}$;
      \STATE $E_{i+1} \leftarrow E_i/\cyc{Q_{e_A-1}}$;
      \FOR{$i\le j < e_A-1$}\label{alg:iso:eval}
      \STATE $Q_{j+1} \leftarrow \phi_i(Q_j)$;
      \ENDFOR
      \ENDFOR
      \ENSURE $E_{e_A}$
    \end{algorithmic}
  \end{minipage}
  
  \caption{Key exchange algorithms.}
  \label{fig:kea}
\end{figure}

A quick analysis shows that both algorithms require $O(\log^2 p)$
operations in $\FF_p$. The major cost in the multiplication-based one
is scalar point multiplication; this costs $O(e_A\log_2\ell_A)$
double-and-adds at each iteration and is repeated
$e_A\sim\log_{\ell_A}\sqrt{p}$ times. The major cost in the
isogeny-based algorithm is the isogeny evaluation at
step~\ref{alg:iso:eval}; each evaluation costs $O(\ell_A)$ operations
and there are $\frac{1}{2}e_A(e_A-1)$ of them. By forming the ratio of
these quantities, we obtain $O(\log_2\ell_A/\ell_A)$, so we see that
the multiplication-based algorithm is preferable as $\ell_A$
grows---but we cannot grow $\ell_A$ indefinitely, because eventually
Step~\ref{alg:mul:velu} becomes the dominant cost. Our implementation,
described in Section~\ref{sec:imp}, supports the isogeny-oriented
approach for $\ell_A = 2,3$ and the multiplication-oriented approach
for $\ell_A > 2$.

\subsection{Isogenies of Montgomery curves}\label{subsec:montgomery}

Independently of which method is chosen, it is important to use pick
models for elliptic curves that offer the fastest isogeny evaluation
performance. The literature on efficient formulas for evaluating small
degree isogenies is much less extensive than for point
multiplication. In this section we provide explicit and efficient
formulas for evaluating isogenies using curves in Montgomery form.

Each of
our curves has group structure $\left(\ZZ/(p\mp 1)\ZZ\right)^2$ and
its twist has group structure $\left(\ZZ/(p\pm 1)\ZZ\right)^2$. Hence either
the curve or its twist has a point of order~$4$. Consequently, we can write
our curves in Montgomery form as follows:
\begin{equation}
  \label{eq:montgomery}
  E\;:\;B^2y^2 = x^3 + Ax^2 + x
\end{equation}
Montgomery curves have very efficient arithmetic on their Kummer line
(i.e.\ by representing points by the coordinates $(X:Z)$ where
$x=X/Z$)~\cite{montgomery}. The Kummer line identifies $P$ with $-P$,
and thus it is not possible to add two distinct points; however it is
still possible to compute any scalar multiple of a point.  Also
observe that since $P$ and $-P$ generate the same subgroup, isogenies
can be defined and evaluated correctly on the Kummer line.  The goal
of this section is to give explicit and efficient formulas for such
isogenies.

Let $E$ be the Montgomery curve defined by Eq.~\eqref{eq:montgomery}.
It has a point of order two in point $P_2=(0,0)$, and a point of order
four in $P_4=(1,\sqrt{(A+2)/B})$---eventualy defined over a quadratic
extension---such that $[2]P_4=P_2$. Montgomery curves have twists of
the form $\tilde{y}=\sqrt{c}y$; these are isomorphisms when $c$ is
a square. The change of coordinates $\tilde{x}=x/B, \tilde{y}=y/B$
brings the curve $E$ to the Weierstrass form
\begin{equation*}
  \tilde{E}\;:\;\tilde{y}^2 = \tilde{x}^3 + \frac{A}{B}\tilde{x}^2 + \frac{1}{B^2}\tilde{x},
\end{equation*}
and the point $P_4$ to $P_4'=(1/B,\ldots)$. Inversely, given a
Weierstrass curve $\tilde{E}$ with equation
$\tilde{y}^2=\tilde{x}^3+a\tilde{x}^2+b\tilde{x}$, together with a
point $P_4=(1/\beta,\ldots)$---with its ordinate possibly lying in a
quadratic extension---such that $[2]P_4=(0,0)$, the change of
variables $\tilde{x}=x/\beta, \tilde{y}=y/\beta$ brings $\tilde{E}$ to
the Montgomery form $\beta y^2=x^3+a\beta x^2 + x$.

Let $G$ be a subgroup of the Montgomery curve $E$ of odd cardinality
$\ell$ and let $h$ be the degree $(\ell-1)/2$ polynomial vanishing on
the abscissas of $G$. With a twist $y=\tilde{y}/\sqrt{B}$, we can put
$E$ in the form $\tilde{y}^2 = \tilde{x}^3 + A\tilde{x}^2 +
\tilde{x}$, and this doesn't change the abscissas of $G$ or the
polynomial $h$. Now we can use V\'elu's formulas to compute an isogeny
having $G$ for kernel: this gives an isogeny $\phi$ and a curve $F$
such that
\begin{align*}
  &F\;:\; y^2 = x^3 + a_2x^2 + a_4x + a_6,\\
  &\begin{aligned}
    \phi : E &\to F,\\
    (x,y) &\mapsto \left(\frac{g(x)}{h(x)^2}, y\sqrt{B}\left(\frac{g(x)}{h(x)^2}\right)'\right).
  \end{aligned}
\end{align*}
Because $\ell$ is odd, the point $(0,0)$ of $E$ is sent to a point of
order two in $F$. A closer look at V\'elu's formulas (see
Eq.~\eqref{eq:bmss} below) shows that $\phi(0,0) = (p_{-1} - p_1, 0)$,
where $p_1$ is the sum of the abscissas of $G$ and $p_{-1}$ is the sum
of their inverses. By the change of variables
$\tilde{x}=x-p_{-1}+p_1$, we bring $F$ to the form
$\tilde{F}:\tilde{y}^2=\tilde{x}^3+a\tilde{x}^2+b\tilde{x}$. Now
$\phi(P_4)$ is a point of order four (possibly in a quadratic
extension). Its abscissa in $\tilde{F}$ is rational and is given by
$1/\beta=g(1)/h(1)-p_{-1}+p_1$, so we apply the change of variables
$\tilde{x}=\bar{x}/\beta,\tilde{y}=\bar{x}/\beta$ to obtain a
Montgomery curve. Finally, we have to twist back the image curve to
obtain a curve isogenous over the base field: the twist $\bar{y} =
y\sqrt{B}$ cancels with the first one and leaves us with
square-root-free formulas. The image curve is
\begin{equation}
  \label{eq:montgomery-image}
  B\beta y^2 = x^3 + a\beta x^2 + x.
\end{equation}

To efficiently evaluate these isogenies (either on the full curve or
on the Kummer line) we use~\cite[Proposition~4.1]{Bostan}, which says:
\begin{equation}
  \label{eq:bmss}
  \frac{g}{h} = \ell x + p_1 - 2(3x^2+2Ax+1)\frac{h'}{h} - 4(x^3+Ax^2+x)\left(\frac{h'}{h}\right).
\end{equation}
To evaluate at a point $(x_0,y_0)$, we compute $h(x_0), h'(x_0),
h''(x_0), h'''(x_0)$; applying Horner's rule, this costs $\sim 2\ell$
multiplications using affine coordinates, or $\sim 3\ell$ using
projective coordinates. Then we inject these values in
Eq.~\eqref{eq:bmss} and in its derivative to evaluate the isogeny,
this only takes a constant number of multiplications (plus one
inversion in affine coordinates). Finally, the image of $(x_0,y_0)$ is
given by \[\left(\beta(g(x_0)/h(x_0)-p_{-1}+p_1),\, \beta
y_0(g/h)'(x_0)\right).\] Note that if the $y$-coordinate is not
needed\footnote{While $x$ coordinates are enough to compute V\'elu's
  isogenies and the image curve, this forces the other party to use
  $y$-coordinate-free formulas for point multiplication.}, we can
avoid computing $h'''(x_0)$, thus saving $\sim\ell/2$
multiplications. Of course, for specific small $\ell$, such as
$\ell=3,5$, it is more convenient to write down the isogeny explicitly
in terms of the kernel points and find optimized formulas.

When $\ell=2$, things are more complex, but in our specific case we
can easily deal with it. The isogeny of $E$ vanishing
$(0,0)$ is readily seen as being
\begin{align}
  \label{eq:isogeny-2}
  &F \;:\;  y^2 = x^3 + \frac{A+6}{B}x^2 + 4\frac{A+2}{B^2}x,\\
  &\begin{aligned}
    \phi : E &\to F,\\
    (x,y) &\mapsto \left(\frac{1}{B}\frac{(x-1)^2}{x}, \frac{1}{B}\left(y - \frac{y}{x^2}\right)\right).
  \end{aligned}
\end{align}
If a point $P_8$ satisfying $[4]P_8=(0,0)$ is known, then $\sqrt{A+2}$
can be computed from the abscissa of $\phi(P_8)$, and $F$ can be put
in Montgomery form as before. The isogeny vanishing on a generic point
of order two $P_2\ne(0,0)$ can be easily computed when a point $P_4$
satisfying $[2]P_4=P_2$ is known: change coordinates to bring $P_2$ in
$(0,0)$, then use the abscissa of $P_4$ to express the resulting curve
in Montgomery form (this is the same technique as above, taking
$\ell=1$); notice that this step needs to be done at most once per key
exchange . When points of order $8$ or $4$ are not available, as in
the last few steps of our setting, ordinary Weierstrass forms yield
formulas that require a few extra multiplications.

We conclude this section with operation counts for the key exchange
algorithms. We write $I,M,S$ for the costs of one inversion,
multiplication and squaring in $\FF_{p^2}$ respectively, and we make
the assumption $S\le M$; we count multiplication by constants as
normal multiplications. For simplicity, we only list quadratic terms
in $e_A$.

\paragraph{Multiplication-based.}
If $P$ is a point on the Kummer line, computing $P$ times an $n$-bit
integer costs $(7M+4S)\log_2n$ (see~\cite{montgomery}). Thus the
cumulative cost of Step~\ref{alg:mul:mul}
is \[\sum_{i=1}^{e_A-1}(7M+4S) \log_2\ell_A^{i} \sim
\frac{1}{2}(7M+4S) (\log_2\sqrt{p})^2\log_{\ell_A}2.\] Doubling a
point on the Kummer line only costs $3M+2S$, and thus the cost for
$\ell_A=2$ drops down to $\frac{1}{2}(3M+2S)(\log_2\sqrt{p})^2$.

\paragraph{Isogeny-based}
The only quadratic term in $e_A$ appears at
Step~\ref{alg:iso:eval}. Since we do not need the $y$ coordinate of
the points involved in this step, we only need the values $h(x_0),
h'(x_0), h''(x_0)$ in order to apply Eq.~\eqref{eq:bmss}. We let
$s=(\ell_A-1)/2$ be the degree of $h$. In affine coordinates, since
$h$ is monic, Horner's rule requires $(3s-6)M$, except when
$s=1,2$. Then, to compute $\beta(g(x_0)/h(x_0)-p_{-1}+p_1)$ we need
$I+8M+2S$. For $\ell_A=3$ the total count drops to $I+6M+2S$, and for
$\ell_A=5$ it is $I+8M+2S$. 

In projective coordinates, we first compute $Z,\ldots,Z^s$ at a cost of
$(s-1)M$. Then, if $h=\sum_ih_iX^{s-i}Z^i$, we compute the monomials
$h_iZ^i$ using $sM$. Finally we compute $h,h',h''$ using three applications
of Horner's rule, costing again $(3s-6)M$ when $s\ne1,2$. The final computation
requires $11M+3S$. For $\ell_A=3$ the total count is $10M+2S$, and for
$\ell_A=5$ it is $14M+3S$. 

The difference between the affine and the projective formulas is
$I-2(s-1)M-S$, so the choice between the two must be done according to
the ratio $I/M$.

Finally for $\ell_A=2$, assuming a point of order $8$ on the domain
curve is known (which will always be the
case, except in the last two iterations), evaluating the $x$ part of
Eq.~\ref{eq:isogeny-2} in projective coordinates and bringing
the result back to a Montgomery curve costs $2M+S$.

There are $e_A(e_A-1)$ isogeny evaluations in the algorithm, so,
assuming that $N$ is the cost of doing one evaluation, the total cost
is about
$\frac{1}{2}e_A^2N=\frac{1}{2}N(\log_2\sqrt{p})^2(\log_{\ell_A}2)^2$. We
summarize the main costs of the two algorithms in Table~\ref{tab:counts}.

\begin{table}[t]
  \centering
  \begin{tabular*}{\textwidth}{@{\extracolsep{\fill}} l | r | r | r | r | r }
    $\ell_A$ & 2 & 3 & 5 & 11 & 19\\
    $\log_{\ell_A}2$ & 1 & 0.63 & 0.43 & 0.29 & 0.23\\
    \hline
    Isogeny & $2M+S$ & $4.0M+0.8S$ & $1.7M+0.5S$ & $2.0M+0.2S$ & $2.4M+0.2S$\\
    Multiplication & $3M+2S$ & $4.4M+2.5S$ & $3.0M+1.7S$ & $2.0M+1.1S$ & $1.6M+0.9S$
  \end{tabular*}
  \smallskip
  \caption{Comparative costs for the multiplication and isogeny based algorithms using projective coordinates. The entries must be multiplied by $\frac{1}{2}(\log_2\sqrt{p})^2$ to obtain the full cost.}
  \label{tab:counts}
\end{table}



%%%%%%%%%%%%%%%%%%%%%%%%%%%%%
\section{Security}\label{sec} 
%%%%%%%%%%%%%%%%%%%%%%%%%%%%%

\subsection{Complexity assumptions and security
  proofs}\label{subsec:proof}

As before, let $p$ be a prime of the form $\ell_A^{e_A}
\ell_B^{e_B}\cdot f \pm 1$, and fix a supersingular curve $E_0$ over
$\FF_{p^2}$ together with bases $\{P_{A},Q_{A}\}$ and
$\{P_{B},Q_{B}\}$ of $E_0[\ell_A^{e_A}]$ and $E_0[\ell_B^{e_B}]$
respectively. In analogy with the case of isogenies over ordinary
elliptic curves, we define the following computational problems,
adapted for the supersingular case:

\begin{problem}[Supersingular Isogeny (SSI) problem] Let $\phi_A
  \colon E_0 \to E_A$ be an isogeny whose kernel is
  $\cyc{[m_A]P_A+[n_A]Q_A}$, where $m_A$ and $n_A$ are chosen at
  random from $\ZZ/\ell_A^{e_A}\ZZ$ and not both divisible by $\ell_A$.
  Given $E_A$ and the values $\phi_A(P_B)$, $\phi_A(Q_B)$, find a
  generator $R_A$ of $\cyc{[m_A]P_A+[n_A]Q_A}$.
\end{problem}

We remark that given a generator $R_A = [m_A]P_A+[n_A]Q_A$, it is easy
to solve for $(m_A,n_A)$, since $E_0$ has smooth order and thus
extended discrete logarithms are easy in $E_0$~\cite{teske-ph}.

\begin{problem}[Supersingular Computational Diffie-Hellman
    (SSCDH) problem] Let $\phi_A \colon E_0 \to E_A$ be an isogeny
  whose kernel is equal to $\cyc{[m_A]P_A+[n_A]Q_A}$, and let $\phi_B \colon
  E_0 \to E_B$ be an isogeny whose kernel is
  $\cyc{[m_B]P_B+[n_B]Q_B}$, where $m_A,n_A$ (respectively $m_B,n_B$)
  are chosen at random from $\ZZ/\ell_A^{e_A}\ZZ$ (respectively
  $\ZZ/\ell_B^{e_B}\ZZ$) and not both divisible by $\ell_A$ (respectively
  $\ell_B$). Given the curves $E_A,$ $E_B$ and the points
  $\phi_A(P_B),$ $\phi_A(Q_B),$ $\phi_B(P_A),$ $\phi_B(Q_A)$, find the
  $j$-invariant of $E_0/\cyc{[m_A]P_A+[n_A]Q_A ,[m_B]P_B+[n_B]Q_B}$.
\end{problem}

\begin{problem}[Supersingular Decision Diffie-Hellman (SSDDH)
    problem] Given a tuple sampled with probability $1/2$ from one of
  the following two distributions:
\begin{itemize} 
\item $(E_A, E_B, \phi_A(P_B), \phi_A(Q_B), \phi_B(P_A), \phi_B(Q_A),
  E_{AB})$, where $E_A,$ $E_B$, $\phi_A(P_B),$ $\phi_A(Q_B),$
  $\phi_B(P_A),$ $\phi_B(Q_A)$ are as in the SSCDH problem and \[E_{AB} \iso
  E_0/\cyc{[m_A]P_A+[n_A]Q_A ,[m_B]P_B+[n_B]Q_B},\]
\item $(E_A, E_B, \phi_A(P_B), \phi_A(Q_B), \phi_B(P_A), \phi_B(Q_A),
  E_C)$, where $E_A,$ $E_B$, $\phi_A(P_B),$ $\phi_A(Q_B),$
  $\phi_B(P_A),$ $\phi_B(Q_A)$ are as in the SSCDH problem and \[E_{C} \iso
  E_0/\cyc{[m_A']P_A+[n_A']Q_A ,[m_B']P_B+[n_B']Q_B},\] where
  $m_A',n_A'$ (respectively $m_B',n_B'$) are chosen at random from
  $\ZZ/\ell_A^{e_A}\ZZ$ (respectively $\ZZ/\ell_B^{e_B}\ZZ$) and not both
  divisible by $\ell_A$ (respectively $\ell_B$),
\end{itemize}
determine from which distribution the triple is sampled.
\end{problem}

We conjecture that these problems are computationally infeasible, in
the sense that for any polynomial-time solver algorithm, the advantage
of the algorithm is a negligible function of the security parameter
$\log p$. The resulting security assumptions are referred to as the
SSI, SSCDH, and SSDDH assumptions, respectively. Using the methods
of Stolbunov~\cite{stolbunov-red}, it is a routine exercise to prove
that the protocols of Section~\ref{sec:kep} are secure under SSDDH:

\begin{theorem}\label{thm:kep-proof}
Under the SSDDH assumption, the key-agreement protocol of
Section~\ref{subsec:kep} is session-key secure in the
authenticated-links adversarial model of Canetti and
Krawczyk~\cite{canetti}.
\end{theorem}
\begin{theorem}\label{thm:pk-proof}
If the SSDDH assumption holds, and the hash function family
$\mathcal{H}$ is entropy-smoothing, then the public-key cryptosystem
of Section~\ref{subsec:pk} is IND-CPA.
\end{theorem}

\opt{full-version}{As an illustration of the proof techniques, we
  provide a proof of Theorem~\ref{thm:kep-proof} in
  Appendix~\ref{sec:kep-proof}.
}

\begin{remark}\label{rem:auth}
As in the ordinary case~\cite{R&S,Stol}, our protocols do not provide
authentication. One possible workaround for the time being
is to use classical public-key authentication schemes in conjunction
with the standard observation~\cite[\S 6.2]{sml09} that the
authentication step only needs to be secure against the adversary at the
time of the initial connection.
\end{remark}

\subsection{Hardness of the underlying
  assumptions}\label{subsec:hardness}

Given an SSI (respectively, SSCDH) solver, it is trivial to solve
SSCDH (respectively, SSDDH). There is of course no known reduction
in the other direction, and given that the corresponding question of
equivalence for
discrete logarithms and Diffie-Hellman has not yet been completely
resolved in all cases, it is reasonable
to assume that the question of equivalence of SSI, SSCDH, and SSDDH
is at least hard to resolve. For the purposes of this discussion, we
will presume that SSI is equivalent to SSDDH.

In the context of cryptography, the problem of computing an isogeny
between isogenous supersingular curves was first considered by
Galbraith~\cite{Gal} in 1999. The first published cryptographic
primitive based on supersingular isogeny graphs is the hash function
proposal of Charles et al.~\cite{CGL}, which remains unbroken to date
(the cryptanalysis of~\cite{quis} applies only to the LPS graph-based
hash function from~\cite{CGL}, and not to the supersingular isogeny
graph-based hash functions). The fastest known algorithm for finding
isogenies between supersingular curves in general takes $O(\sqrt{p}
\log^2 p)$ time~\cite[\S 5.3.1]{CGL}; however our problem is less
general because the degree of the isogeny is known in advance and is
smooth. \opt{full-version}{In addition, the distribution of isogenous
  curves obtained
from taking kernels of the form $\cyc{[m_A]P_A+[n_A]Q_A}$ is not quite
uniform: a simple calculation against Proposition~\ref{prop:mixing}
indicates that a sequence of $e_A$ isogenies of degree $\ell_A$ falls
short of the length needed to ensure uniform mixing, regardless of the
value of $p$.} Since we are the first to propose using isogenies of
this type, there is no existing literature addressing the security
of the isogenies of the special form that we propose.

There is an easy exponential attack against our cryptosystem that
improves upon exhaustive search. To find an isogeny of degree
$\ell_A^{e_A}$ between $E$ and $E_A$, an attacker builds two trees of all
curves isogenous to $E$ (respectively, $E_A$) via isogenies of degree
$\ell_A^{e_A/2}$. Once the trees are built, the attacker tries to find
a curve lying in both trees. Since the degree of the isogeny $\phi_A$ is
$\sim\sqrt{p}$ (much shorter than the size of the isogeny graph), it
is unlikely that there will be more than one isogeny path---and thus
more than one match---from $E$ to $E_A$. Given two functions $f:A\to
C$ and $g:B\to C$ with domain of equal size, finding a pair $(a,b)$
such that $f(a)=g(b)$ is known as the \emph{claw problem} in
complexity theory. The claw problem can obviously be solved
in $O(|A| + |B|)$ time and $O(|A|)$ space on a classical computer by
building a
hash table holding $f(a)$ for any $a\in A$ and looking for hits for
$g(b)$ where $b\in B$. This gives a $O(\ell_A^{e_A/2})=O(\sqrt[4]{p})$
classical attack against our cryptosystem. With a quantum computer, one
can do better using the algorithm in~\cite{tani}, which has complexity
$O(\sqrt[3]{|A||B|})$, thus giving an
$O(\ell_A^{e_A/3})=O(\sqrt[6]{p})$ quantum attack against our
cryptosystem. These complexities are optimal for a black-box claw
attack~\cite{zhang}.

We consider the question of whether the auxiliary data points
$\phi_A(P_B)$ and $\phi_A(Q_B)$ might assist an adversary in
determining $\phi_A$. Since $(P_B,Q_B)$ forms a basis for
$E_0[\ell_B^{e_B}]$, the values $\phi_A(P_B)$ and $\phi_A(Q_B)$ allow
the adversary to compute $\phi_A$ on all of $E_0[\ell_B^{e_B}]$. This
is because any element of $E_0[\ell_B^{e_B}]$ is a (known) linear
combination of $P_B$ and $Q_B$ (known since extended discrete
logarithms are easy~\cite{teske-ph}). However, there does not appear
to be any way to use this capability to determine $\phi_A$. Even on a
quantum computer, where finding abelian hidden subgroups is easy,
there is no hidden subgroup to find, since $\phi_A$ has degree
$\ell_A^{e_A}$, and thus does not annihilate any point in
$E_0[\ell_B^{e_B}]$ other than the identity. Of course, if one could
evaluate $\phi_A$ on arbitrary points of $E_0[\ell_A^{e_A}]$, then a
quantum computer could easily break the scheme, and indeed in this case the
scheme is also easily broken classically by using a few calls to the
oracle to compute a generator of the kernel of the dual isogeny 
$\hat{\phi}_A$. However, it does not seem possible to translate the values of
$\phi_A$ on $E_0[\ell_B^{e_B}]$ into values on $E_0[\ell_A^{e_A}]$.

Finally, we discuss the possibility of adapting the quantum algorithm
of Childs et al.~\cite{CJS} for the ordinary case to the supersingular
case. For both ordinary and supersingular curves, there is a natural
bijection between isogenies (up to isomorphism) and (left) ideal
classes in the endomorphism ring. The algorithm of Childs et
al. depends crucially on the fact that the ideal classes in the
ordinary case form an abelian group. In the supersingular case, the
endomorphism ring is a maximal order in a noncommutative quaternion
algebra, and the left ideal classes do not form a group at all
(multiplication is not well defined). Thus we believe that no
reasonable variant of this strategy would apply to supersingular
curves.

\section{Implementation results and example}\label{sec:imp}

We implemented our cryptosystem in the computer algebra system
Sage~\cite{Sage} using a mixed C/Cython/Sage architecture. This allows
us to access the large palette of number theoretic algorithms
distributed with Sage, while still permitting very efficient code in
C/Cython for the critical parts such as the algorithms of
Section~\ref{subsec:kea}. The source code can be downloaded from the
second author's web page.

Arithmetic in $\FF_{p^2}$ is written in C. We use the library GMP for
arithmetic modulo $p$. The field $\FF_{p^2}$ is implemented as
$\FF_{p^2}[X]/(X^2+1)$ (this requires $p=3\bmod4$); using this
representation, one multiplication in $\FF_{p^2}$ requires three
multiplications ($3M$) in $\FF_p$, one $\FF_{p^2}$ squaring requires
two multiplications ($2M$) in $\FF_p$, and one $\FF_{p^2}$ inversion
requires one inversion, two squarings, and two multiplications
($I+2S+2M$) in $\FF_p$. Our experiments show that, for the sizes we
are interested in, $I=10M$ and $S=0.8M$.

We implemented the isogeny-based key exchange algorithm for $\ell=2,3$
and the multiplication-based algorithm for $\ell>2$.  The main loop is
implemented in Cython, while the isogeny evaluation and the Montgomery
ladder formulas are written in C.

Finally, the parameter generation is implemented in plain
Sage. Because Sage is a collection of many open source mathematical
systems, various of its subsystems are involved in this last part. Of
these, Pari~\cite{Pari} plays an important role because it is used to
compute Hilbert class polynomials and to factor polynomials over
finite fields.

\begin{table}[t]
  \centering
  \begin{tabular}{l | r | r | r | r}
    & \multicolumn{2}{c|}{Alice} & \multicolumn{2}{|c}{Bob}\\
    & round 1 & round 2 & round 1 & round 2\\
    \hline
    $2^{253}3^{161}7-1$ &  365 ms &  363 ms &  318 ms &  314 ms \\
    $5^{110} 7^{91} 284 - 1$ & 419 ms & 374 ms & 369 ms & 326 ms \\
    $11^{74}13^{69}384-1$ & 332 ms & 283 ms & 321 ms & 272 ms \\
    $17^{62}19^{60}210+1$ & 330 ms & 274 ms & 331 ms & 276 ms\\
    $23^{56}29^{52}286+1$ & 339 ms & 274 ms & 347 ms & 277 ms \\
    $31^{51}41^{47}564-1$ & 355 ms & 279 ms & 381 ms & 294 ms \\
    \hline
    $2^{384}3^{242}8-1$ & 1160 ms & 1160 ms &  986 ms &  973 ms \\
    $5^{165}7^{137}2968-1$ & 1050 ms & 972 ms & 916 ms & 843 ms\\
    $11^{111}13^{104} 78 + 1$ & 790 ms & 710 ms & 771 ms & 688 ms\\
    $17^{94}19^{90}116 - 1$ & 761 ms & 673 ms & 750 ms & 661 ms\\
    $23^{85}29^{79}132 - 1$ & 755 ms & 652 ms & 758 ms & 647 ms\\
    $31^{77}41^{72}166 + 1$ & 772 ms & 643 ms & 824 ms & 682 ms\\
    \hline
    $2^{512}3^{323}799-1$ & 2570 ms & 2550 ms & 2170 ms & 2150 ms\\
    $5^{220}7^{182}538 + 1$ & 2270 ms & 2140 ms & 1930 ms & 1810 ms \\
    $11^{148}13^{138}942 + 1$ & 1650 ms & 1520 ms & 1570 ms & 1440 ms\\
    $17^{125}19^{120}712 - 1$ & 1550 ms & 1430 ms & 1520 ms & 1380 ms\\
    $23^{113}29^{105}1004-1$ & 1480 ms & 1330 ms & 1470 ms & 1300 ms
  \end{tabular} 

\vspace{1em}

  \caption{Benchmarks for various group sizes and structures.}
  \label{tab:benchs}
\end{table}

All tests ran on a 2.4 GHz Opteron running in 64-bit mode. The results are
summarized in Table~\ref{tab:benchs}. At the quantum 128-bit security level
(768-bit $p$), our numbers improve upon Stolbunov's
reported performance
figures~\cite[Table 1]{Stol} by over two orders of magnitude (.758
seconds vs. 229 seconds).
This is the highest security level appearing in~\cite[Table 1]{Stol}, so
comparisons at higher levels are difficult. Nevertheless, it
seems safe to assume that the improvement is even greater at the
256-bit security level. Our results demonstrate that the proposed
scheme is practical.

\subsection{Example}\label{sec:ex}

As a convenience, we provide an example calculation of a key-exchange
transaction. Let $\ell_A = 2$, $\ell_B = 3$, $e_A = 63$, $e_B = 41$,
and $f = 11$. We use the starting curve $E_0: y^2 = x^3 + x$. For the
torsion bases, we use {\tiny
\begin{align*}
P_A &= (2374093068336250774107936421407893885897 i + 
    2524646701852396349308425328218203569693, \\
    & \qquad 1944869260414574206229153243510104781725 i + 
    1309099413211767078055232768460483417201) \\
P_B &= (1556716033657530876728525059284431761206 i + 
    1747407329595165241335131647929866065215, \\
    & \qquad 3456956202852028835529419995475915388483 i + 
    1975912874247458572654720717155755005566)
\end{align*}
}
{\!\!\!} and $Q_A = \psi(P_A), Q_B = \psi(P_B)$, where $i = \sqrt{-1}$ in
$\FF_{p^2}$ and $\psi(x,y) = (-x,iy)$ is a distortion map~\cite{joux}. The
secret values are
{\tiny
\begin{align*}
m_A &= 2575042839726612324,\ 
n_A = 8801426132580632841,\  \\
m_B &= 4558164392438856871,\ 
n_B = 20473135767366569910 
\end{align*}
}
The isogeny $\phi_A\colon E_0 \to E_A$ is specified by its kernel, and
thus the curve $E_A$ is only well defined up to isomorphism; its exact
value may vary depending on the implementation. In our case, the curve
is $E_A: y^2 = x^3 + ax + b$ where
{\tiny
\begin{align*}
a &= 428128245356224894562061821180718114127 i + 2147708009907711790134986624604674525769 \\
b &= 3230359267202197460304331835170424053093 i + 1577264336482370197045362359104894884862
\end{align*}
}
and the values of $\phi_A(P_B)$ and $\phi_A(Q_B)$ are
{\tiny
\begin{align*}
\phi_A(P_B) &= 
(1216243037955078292900974859441066026976 i + 
    1666291136804738684832637187674330905572, \\
    & \qquad 3132921609453998361853372941893500107923 i + 
    28231649385735494856198000346168552366)
\\
\phi_A(Q_B) &=
(2039728694420930519155732965018291910660 i + 
    2422092614322988112492931615528155727388, \\
    & \qquad 1688115812694355145549889238510457034272 i + 
    1379185984608240638912948890349738467536)
\end{align*}
}
Similarly, in our implementation $E_B : y^2 = x^3 + ax
+ b$ is the curve with
{\tiny
\begin{align*}
a &= 2574722398094022968578313861884608943122 i + 464507557149559062184174132571647427722 \\
b &= 2863478907513088792144998311229772886197 i + 1767078036714109405796777065089868386753
\end{align*}
}
\opt{full-version}{and the values of $\phi_B(P_A)$ and $\phi_B(Q_A)$ are
{\tiny
\begin{align*}
\phi_B(P_A) &=
(2519086003347973214770499154162540098181 i + 
    1459702974009609198723981125457548440872, \\
    & \qquad 2072057067933292599326928766255155081380 i + 
    891622100638258849401618552145232311395)
\\
\phi_B(Q_A) &=
(53793994522803393243921432982798543666 i + 
    3698741609788138685588489568343190504844, \\
    & \qquad 2853868073971808398649663652161215323750 i + 
    1869730480053624141372373282795858691139)
\end{align*}
}}The common $j$-invariant of $E_{AB} \iso E_{BA}$, computed by both
Alice and Bob, is equal to
{\tiny
\[
j(E_{AB}) = 1437145494362655119168482808702111413744 i +
833498096778386452951722285310592056351.\]
}
\vspace{-2em}

%%%%%%%%%%%%%%%%%%%%%%%%%%%%%
\section{Conclusion}
%%%%%%%%%%%%%%%%%%%%%%%%%%%%%

We propose a new family of conjecturally quantum-resistant cryptographic
protocols for key exchange and public-key cryptosystems using isogenies
between supersingular elliptic curves of smooth order. In order to
compensate for the noncommutative endomorphism rings that arise in this
setting, we introduce the idea of providing the images of torsion bases
as part of the protocol. Against the fastest known attacks, the
resulting scheme improves upon all previous isogeny-based schemes by
orders of magnitude in performance at conventional security levels,
making it the first practical isogeny-based public-key cryptosystem.
Unlike prior such schemes, our proposal admits no known
subexponential-time attacks even in the quantum setting.

\subsection*{Acknowledgements}

We thank Andrew M. Childs, Alfred Menezes, Vladimir Soukharev, and
the anonymous reviewers for helpful comments and suggestions. This work
is supported in part by NSERC CRD Grant CRDPJ 405857-10.

\bibliographystyle{plain}
\opt{conference}{
\bibliography{crypto}
}
\opt{full-version}{
\bibliography{crypto-long}
}

\opt{full-version}{
%%%%%%%%%%%%%%%%%%%%%%%%%%%%%
\appendix
%%%%%%%%%%%%%%%%%%%%%%%%%%%%%
\section{Proof of Theorem~\ref{thm:kep-proof}}\label{sec:kep-proof}
%%%%%%%%%%%%%%%%%%%%%%%%%%%%%

We recall the definition of session-key security in the
authenticated-links adversarial model of Canetti and
Krawczyk~\cite{canetti}. We consider a finite set of \emph{parties}
$P_1, P_2, \ldots, P_n$ modeled by probabilistic Turing machines.  The
adversary $\mathcal{I}$, also modeled by a probabilistic Turing
machine, controls all communication, with the exception that the
adversary cannot inject or modify messages (except for messages from
corrupted parties or sessions), and any message may be delivered at
most once. Parties give outgoing messages to the adversary, who has
control over their delivery via the \textsf{Send} query. Parties are
activated by \textsf{Send} queries, so the adversary has control over
the creation of protocol sessions, which take place within each party.
Two sessions $s$ and $s'$ are \emph{matching} if the outgoing messages
of one are the incoming messages of the other, and vice versa.

We allow the adversary access to the queries
$\mathsf{SessionStateReveal}$, $\mathsf{SessionKeyReveal}$, and
$\mathsf{Corrupt}$. The $\mathsf{SessionStateReveal}(\mathfrak{s})$
query allows the adversary to obtain the contents of the session
state, including any secret information. The query is noted and
$\mathfrak{s}$ produces no further output.  The
$\mathsf{SessionKeyReveal(\mathfrak{s})}$ query enables the adversary
to obtain the session key for the specified session $\mathfrak{s}$, so
long as $\mathfrak{s}$ holds a session key. The
$\mathsf{Corrupt(P_i)}$ query allows the adversary to take over the
party $P_i$, i.e.,\ the adversary has access to all information in
$P_i$'s memory, including long-lived keys and any session-specific
information still stored. A corrupted party produces no further
output. We say a session $\mathfrak{s}$ with owner $P_i$ is
\emph{locally exposed} if the adversary has issued
$\mathsf{SessionKeyReveal(\mathfrak{s})}$,
$\mathsf{SessionStateReveal(\mathfrak{s})}$, or
$\mathsf{Corrupt(P_i)}$ before $\mathfrak{s}$ is expired. We say
$\mathfrak{s}$ is \emph{exposed} if $\mathfrak{s}$ or its matching
session have been locally exposed, and otherwise we say $\mathfrak{s}$
is \emph{fresh}.

We allow the adversary $\mathcal{I}$ a single
$\mathsf{Test(\mathfrak{s})}$ query, which can be issued at any stage
to a completed, fresh, unexpired session $\mathfrak{s}$. A bit $b$ is
then picked at random. If $b=0$, the test oracle reveals the session
key, and if $b=1$, it generates a random value in the key
space. $\mathcal{I}$ can then continue to issue queries as desired,
with the exception that it cannot expose the test session. At any
point, the adversary can try to guess $b$. Let
$\operatorname{GoodGuess}^{\mathcal{I}}(k)$ be the event that
$\mathcal{I}$ correctly guesses $b$, and define \[
\operatorname{Advantage}^{\mathcal{I}}(k) = \max \left\{0, \left \vert
\operatorname{Pr}[\operatorname{GoodGuess}^{\mathcal{I}}(k)] - \frac 1
2 \right \vert \right\},\] where $k$ is a security parameter.

The definition of security is as follows:

\begin{definition}\label{def:kep}
A key exchange protocol $\Pi$ in security parameter $k$ is said to be
\emph{session-key secure} in the authenticated-links adversarial model
of Canetti and Krawczyk if for any polynomial-time adversary $\mathcal{I}$,
\begin{enumerate}
\item If two uncorrupted parties have completed matching sessions,
  these sessions produce the same key as output;
\item $\operatorname{Advantage}^{\mathcal{I}}(k)$ is negligible.
\end{enumerate}
\end{definition}

\begin{algorithm}[t]
\caption{SSDDH distinguisher}
\label{alg:distinguisher}
\begin{algorithmic}[1]
\REQUIRE $E_A, E_B, \phi_A(P_B), \phi_A(Q_B), \phi_B(P_A),
\phi_B(Q_A), E$
\STATE $r \stackrel{R}{\leftarrow} \{1,\ldots,k\}$, where $k$ is an
upper bound on the number of sessions activated by $\mathcal{I}$ in any
interaction.
\STATE Invoke $\mathcal{I}$ and simulate the protocol to $\mathcal{I}$, except for the
$r$-th activated protocol session.
\STATE For the $r$-th session, let Alice send $A, i, E_A, \phi_A(P_B),
\phi_A(Q_B)$ to Bob, and let Bob send $B, i, E_B, \phi_B(P_A),
\phi_B(Q_A)$ to Alice, where $i$ is the session identifier.
\IF{the $r$-th session is chosen by $\mathcal{I}$ as the test session}
\STATE Provide $\mathcal{I}$ as the answer to the test query.
\STATE $d \leftarrow \mathcal{I}$'s output.
\ELSE
\STATE $d \stackrel{R}{\leftarrow}\{0,1\}$.
\ENDIF
\ENSURE $d$

\end{algorithmic}
\end{algorithm}

\begin{theorem}
Under the SSDDH assumption, the key-agreement protocol of
Section~\ref{subsec:kep} is session-key secure in the
authenticated-links adversarial model of Canetti and
Krawczyk~\cite{canetti}.
\end{theorem}

\begin{proof}
The proof is based on the proof given by Canetti and Krawczyk~\cite[\S
  5.1]{canetti} for two-party Diffie-Hellman over $\ZZ_q^*$. A similar
strategy was used by Stolbunov~\cite{stolbunov-red} in the case of
ordinary elliptic curves.

It has been shown in Section~\ref{sec:kep} that two uncorrupted
parties in matching sessions output the same session key, and thus the
first part of Definition~\ref{def:kep} is satisfied. To show that the
second part of the definition is satisfied, assume that there is a
polynomial-time adversary $\mathcal{I}$ with a non-negligible advantage
$\varepsilon$. We claim that Algorithm~\ref{alg:distinguisher} forms a
polynomial-time distinguisher for SSDDH having non-negligible
advantage.

To prove the claim, we must show that
Algorithm~\ref{alg:distinguisher} has non-negligible advantage (it is
clear that it runs in polynomial time). We consider separately the
cases where the $r$-th session is (respectively, is not) chosen by
$\mathcal{I}$ as the test session. If the $r$-th session is not the
test session, then Algorithm~\ref{alg:distinguisher} outputs a random
bit, and thus its advantage in solving the SSDDH is $0$. If the
$r$-th session is the test session, then $\mathcal{I}$ will succeed
with advantage $\varepsilon$, since the simulated protocol provided to
$\mathcal{I}$ is indistinguishable from the real protocol. Since the
latter case occurs with probability $1/k$, the overall advantage of
the SSDDH distinguisher is $\varepsilon/k$, which is non-negligible.
\end{proof}

}

\end{document}

